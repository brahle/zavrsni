\documentclass[times, utf8, zavrsni]{fer}
\usepackage{booktabs}
\usepackage{xcolor}
\usepackage{colortbl}
\usepackage{hyperref}
\hypersetup{
unicode=true, % za unicode podršku u bookmarkovima i sl., svakako postaviti na true
colorlinks=true, % za prikaz linkova u boji umjesto u pravokutnicima; slijede postavke po klasama linkova
linkcolor=black, % možete koristiti ime bilo koje definirane boje (pročitati post u kojem je predstavljen paket xcolor)
citecolor=black,
filecolor=black,
urlcolor=black,
linktocpage=false, % ako želite da oznaka stranice u tablici sadržaja bude link umjesto naslova, postavite na true
bookmarksopen=true, % pri otvaranju dokumenta će se prikazati toolbar s bookmarkovima, jako korisno
bookmarksopenlevel=1, % bookmarkovi su prikazani u stablastoj strukturi, a ovim određujete do koje će dubine stablo bookmarkova biti razgranato
bookmarksnumbered=true, % uz imena bookmarkova će se prikazati i njihov broj
pdftitle={SWIG - Poravnanje struktura korištenjem iterativne primjene Smith-Waterman algoritma},
pdfauthor={Bruno Rahle},
pdfsubject={tema},
pdfkeywords={kljucne rijeci - odvojite ih razmakom},
pdfstartpage={}, % ovisno koji stil numeriranja stranica koristite (\Roman, \roman, \arabic, \alph, \Alph), možete postaviti broj (oznaku) stranice koja će se prikazati prilikom otvaranja dokumenta; prazne vitičaste zagrade uvijek znače prvu stranicu, bez obzira na koji način bila numerirana
pdfstartview=FitH, % dokument će se prikazati tako da će mu širina odgovarati širini PDF preglednika
}
\def\subsectionautorefname{odlomak}
\def\sectionautorefname{potpoglavlje}
\def\chapterautorefname{poglavlje}

\begin{document}

% TODO: Navedite broj rada.
\thesisnumber{2321}

% TODO: Navedite naslov rada.
\title{SWIG - Poravnanje struktura korištenjem iterativne primjene Smith-Waterman algoritma}

% TODO: Navedite vaše ime i prezime.
\author{Bruno Rahle}

\maketitle

% Ispis stranice s napomenom o umetanju izvornika rada. Uklonite naredbu \izvornik ako želite izbaciti tu stranicu.
\izvornik

% Dodavanje zahvale ili prazne stranice. Ako ne želite dodati zahvalu, naredbu ostavite radi prazne stranice.
\zahvala{}

\tableofcontents

\listoffigures

\listoftables

\chapter{Uvod}
Problemi s kojima se znanost danas susreće često prelaze granice isključivo
jedne discipline. 

- pričaj o: bioinformatici, koje probleme ona rješava, zašto je problem koji si rješavao bitan,
kome koristi, kako se može koristiti i potencijalno za što se sve može koristiti.
- napiši zašto je ovo što si napravio jebeno i kako se može još poboljšati.
- cca 2 stranice

\chapter{Poravanavanje sturktura}
- detaljan opis problema
- cca 1-2 stranice

\chapter{Smith-Watermanov algoritam}
Smith-Watermanov algoritam služi nam da bismo pronašli lokalno poravnanje. 
Osmislili su ga Temple F. Smith i Michael S. Waterman 1981. \cite{smithwaterman1981}.
Temelji se na Needleman-Wunschevom algoritmu (\cite{needlemanwunsch1970})
te i sam spada u kategoriju algoritama dinamičkog programiranja. Glavna razlika
između ta dva algoritma jest što Needleman-Wunschov algoritam prolazi globalno
poravananje.

- napisi jos teksta ovdje.

\section{Needleman-Wunschov algoritam}
\label{sec:nwalg}
Algoritam, kao što je već rečeno, traži globalno poravnanje. To znači da se svi 
članovi ulaznih nizova moraju poravnati. Dopuštene operacije kada tražimo poravnanje
su preklapanje s elementom iz suprotnog niza i ubacivanje praznina u neki od nizova.
Sve parove elemenata u dobivenom preklapanju bodujemo i na osnovu te ocjene
određujemo sličnost nizova. Konačan rezultat ovog algoritma jest poravnanje koje
maksimizira takvu ocjenu, tj. daje maksimalno globalno poravnanje.

Zanimljivost je da je to prvi algoritam dinamičkog programiranja ikada primjenjen
u bioinformatici.

\subsection{Ulazni podaci}
\label{subsec:nwinp}
\begin{enumerate}
	\item Dva niza ($A$ i $B$) proteina ili nukleotida. Zbog jednostavnosti, pretpostavit
		ćemo da su to nukleotidi iz DNK - adenin (A), timin (T), gvanin (G) i citozin (C).
		U primjeru ćemo koristiti $A = "ATGCCGTA"$ i $B = "TGCACTA"$. Dužinu niza $A$
		označit ćemo s $N$, a dužinu niza $B$ s $M$. 
	\item Supstitucijska matrica $S$, koja nam daje bodove koje dobijemo kada jedan
		nukletoid preklopimo s drugim. U našem će slučaju imati
		dimenzije $4 \times 4$, budući da ćemo razmatrati
		slučaj kada imamo samo četiri nukleotida. U principu će na dijagonali
		imati pozitivne brojeve, a na ostalim poljima negativne. To znači da nam
		se najviše isplati preklapati nukletide istog tipa, jer za to dobivamo bodove,
		a inače ih	gubimo. Primjer jedne takve matrice koju ćemo koristiti i u primjeru:
		
		$\begin{tabular}{c|cccc}
		& A & C & G & T \\\specialrule{1pt}{0pt}{0pt}
		A & 10 & -3 & -9 & -1 \\
		C & -5 & 8 & -8 & -7 \\
		G & -5 & -4 & 7 & -5\\
		T & -4 & -11 & -8 & 9
		\end{tabular}
		$
	\item Negativan broj $d$, koji označava bodove koje dobijemo (tj. izgubimo) kada nukleotid
		preklopimo s prazninom. U primjeru ćemo koristiti $d = -5$.
\end{enumerate}

\subsection{Izlazni podaci}
\label{subsec:nwout}
\begin{enumerate}
	\item Broj $H$, ocjena najboljeg globalnog poravnanja.
	\item Dva nova niza jednake dužine, $A'$ i $B'$, nastala ubacivanjem praznina
		(označenih najčešće sa '-') u nizove $A$ i $B$ koja predstavljaju najbolje pronađeno
		poravnanje.
\end{enumerate}

\subsection{Algoritam}
\label{subsec:nwalg}
Neka nam matrica $F$ služi za računanje poravnanja. 
Tada će nam $F_{i,j}$ označavati maksimalan broj bodova koje možemo dobiti kada
poravnamo prvih $i$ članova niza $A$ i prvih $j$ članova niza $B$. $F_{i,j}$ možemo
računati rekurzijom na slijedeći način:
$$
F_{i,j} = \left\{
	\begin{array}{lr}
		0 & \mbox{ako je } i=0 \mbox{ i } j=0 \\
		F_{i-1,j} + d & \mbox{ako je } i>0 \mbox{ i } j=0 \\
		F_{i,j-1} + d & \mbox{ako je } i=0 \mbox{ i } j>0 \\
		max \left(
			\begin{array}{l}
				F_{i-1,j-1} + S_{A_{i-1}, B_{j-1}} \\
				F_{i-1, j} + d \\
				F_{i, j-1} + d
			\end{array}
		\right) & \mbox{ako je } i>0 \mbox{ i } j>0
	\end{array}
\right.
$$

U $F_{N,M}$ će nam stoga pisati maksimalno globalno poravnanje. Primjetite da
u niti jednom slučaju nećemo poravnati dvije praznine. Ako bismo to učinili, 
samo bismo izgubili bodove, budući da je $d$ nužno negativan broj.

Da bismo znali rekonstruirati rješenje, koristit ćemo matricu $R$. U
polju $R_{i,j}$  pisat će koje smo polje matrice $F$ koristili da bi došli u
polje $F_{i,j}$. Kako su jedine mogućnosti $F_{i-1,j}$, $F_{i,j-1}$, $F_{i-1,j-1}$
i da nismo došli iz nikojeg polja (to vrijedi jedino za polje $F_{0,0}$), koristit
ćemo redom oznake $A$, $B$, $O$ i $X$. 

$$
R_{i,j} = \left\{
	\begin{array}{ll}
		X & \mbox{ako je} \left( \begin{array}{l} i=0 \\ j=0 \end{array} \right) \\
		O & \mbox{ako je} \left( \begin{array}{l} i>0 \\ j>0 \\
				F_{i-1,j-1} + S_{A_{i-1}, B_{j-1}} \geq F_{i-1,j} + d \\
				F_{i-1,j-1} + S_{A_{i-1}, B_{j-1}} \geq F_{i,j-1} + d \\
			\end{array} \right) \\
		A & \mbox{ako je} \left( \begin{array}{l} i>0 \\ j=0 \end{array} \right)
			\mbox{ili} \left( \begin{array}{l} i>0 \\ j>0 \\
				F_{i-1,j} + d > F_{i-1,j-1} + S_{A_{i-1}, B_{j-1}} \\
				F_{i-1,j} + d \geq F_{i,j-1} + d
			\end{array} \right) \\
		B & \mbox{ako je} \left( \begin{array}{l} i=0 \\ j>0 \end{array} \right)
			\mbox{ili} \left( \begin{array}{l} i>0 \\ j>0 \\
				F_{i,j-1} + d > F_{i-1,j-1} + S_{A_{i-1}, B_{j-1}} \\
				F_{i,j-1} + d > F_{i-1,j} + d
			\end{array} \right) \\
	\end{array}
\right.
$$

Rekonstrukciju provodimo tako krenemo iz polja $R_{N,M}$ i krećemo se
po matrici unazad dok ne dođemo do polja na kojem piđe $X$, tj. $R_{0,0}$.
Ako na polju pročitamo $O$,
pomičemo se po dijagonili, tj. u izlazni niz spremimo par $(A_{i-1}, B_{j-1})$ te
smanjimo i $i$ i $j$. Ako pročitamo $A$, spremamo par $(A_{i-1}, -)$ te smanjimo
samo $i$ za jedan. U slučaju da pročitamo $B$, spremamo par $(-, B_{j-1})$ te
smanjujemo $j$ za jedan. Ako smo pročitali $X$, došli smo do kraja i generirali
smo izlazni niz, ali u obrnutom redosljedu. 

\subsection{Primjer}
Za prethodno navedene ulazne podatke, matrice $F$ i $R$ izgledat će ovako:
$$
F =
\begin{tabular}{c|cccccccc}
 & - & T & G & C & A & C & T & A \\\specialrule{0.5pt}{0pt}{0pt}
- & \cellcolor{lightgray} 0 & -5 & -10 & -15 & -20 & -25 & -30 & -35 \\ 
A & \cellcolor{lightgray} -5 & -1 & -6 & -11 & -5 & -10 & -15 & -20 \\ 
T & -10 & \cellcolor{lightgray} 4 & -1 & -6 & -10 & -15 & -1 & -6 \\ 
G & -15 & -1 & \cellcolor{lightgray} 11 & 6 & 1 & -4 & -6 & -6 \\ 
C & -20 & -6 & 6 & \cellcolor{lightgray} 19 & \cellcolor{lightgray} 14 & 9 & 4 & -1 \\ 
C & -25 & -11 & 1 & 14 & 14 & \cellcolor{lightgray} 22 & 17 & 12 \\ 
G & -30 & -16 & -4 & 9 & 9 & \cellcolor{lightgray} 17 & 17 & 12 \\ 
T & -35 & -21 & -9 & 4 & 5 & 12 & \cellcolor{lightgray} 26 & 21 \\ 
A & -40 & -26 & -14 & -1 & 14 & 9 & 21 & \cellcolor{lightgray} 36 \\ 
\end{tabular}
$$

$$
R =
\begin{tabular}{c|cccccccc}
 & - & T & G & C & A & C & T & A \\\specialrule{0.5pt}{0pt}{0pt}
- & \cellcolor{lightgray} X & B & B & B & B & B & B & B \\ 
A & \cellcolor{lightgray} A & O & B & B & O & B & B & O \\ 
T & A & \cellcolor{lightgray} O & B & B & A & A & O & B \\ 
G & A & A & \cellcolor{lightgray} O & B & B & B & A & O \\ 
C & A & A & A & \cellcolor{lightgray} O & \cellcolor{lightgray} B & O & B & B \\ 
C & A & A & A & O & O & \cellcolor{lightgray} O & B & B \\ 
G & A & A & O & A & O & \cellcolor{lightgray} A & O & O \\ 
T & A & O & A & A & O & A & \cellcolor{lightgray} O & B \\ 
A & A & A & A & A & O & B & A & \cellcolor{lightgray} O \\ 
\end{tabular}
$$

Iz tih podataka lagano je napraviti rekonstrukciju (polja označena sivom
bojom). Stoga zaključujemo da je traženo globlano poravnanje
$A' = \mbox{"ATGCCGTA"}$ i $B' = \mbox{"-TGCACTA"}$.

\subsection{Analiza složenosti}
Trivijalno je vidljivo da su memorijska i vremenska složenost opisanog algoritma 
jednake $O(NM)$. 

\section{Smith-Watermanov algoritam}
\label{sec:swalg}
Razlika njega i prethodno opisanog Needleman-Wunschevog algoritma jest u tome
što ovaj algoritam traži najbolje lokalno poravnanje. To znači da ne koristi
nužno cijele nizove proteina ili nukleotida već samo najsličnije uzastopne
podnizove. U praksi se koriste nešto poboljšane verzije ovog algoritma.

\subsection{Ulazni podaci}
Vidi \autoref{subsec:nwinp}.

\subsection{Izlazni podaci}
Vidi \autoref{subsec:nwout}.

\subsection{Algoritam}
\label{subsec:swalg}
U ovom ćemo algoritmu koristiti matricu $H$ za računanje poravnanja.
Za razliku od Needleman-Wunschevog algoritma, $H_{i,j}$ ovaj će put
označavati rezultat
najboljeg lokalnog poravnanja koje koristi $A_{i-1}$ i $B_{j-1}$.
Matricu ćemo popuniti na slijedeći način:

$$
H_{i,j} =
\left\{ \begin{array}{ll}
	0 & \mbox{ako je } i=0 \mbox{ ili } j=0 \\
	\max \left( \begin{array}{l}
		0 \\
		H_{i-1,j-1} + S_{A_{i-1}, B_{j-1}} \\
		H_{i-1, j} + d \\
		H_{i, j-1} + d
	\end{array} \right) & \mbox{ako je } i>0 \mbox{ i } j>0
\end{array} \right.
$$

Primjetite bitnu razliku u odnosu na Needleman-Wunschev algoritam - u ovom
slučaju matrica $H$ neće sadržavati negativne brojeve. Rješenje, tj. najbolje
lokalno poravnanje više neće pisati na polju $H_{N, M}$. Ono će biti na polju 
$(t, u)$ na kojem se nalazi najveći broj u matrici.

Da bismo znali rekonstruirati takvo lokalno poravnanje, koristit ćemo matricu
$R$ koju gradimo na sličan način kao i kod prethodnog algoritma:

$$
R_{i,j} =  \left\{
	\begin{array}{ll}
		X & \mbox{ako je} H_{i,j}=0 \\
		O & \mbox{ako je} \left( \begin{array}{l} i>0 \\ j>0 \\
				H_{i-1,j-1} + S_{A_{i-1}, B_{j-1}} \geq H_{i-1,j} + d \\
				H_{i-1,j-1} + S_{A_{i-1}, B_{j-1}} \geq H_{i,j-1} + d \\
			\end{array} \right) \\
		A & \mbox{ako je} \left( \begin{array}{l} i>0 \\ j=0 \end{array} \right)
			\mbox{ili} \left( \begin{array}{l} i>0 \\ j>0 \\
				H_{i-1,j} + d > H_{i-1,j-1} + S_{A_{i-1}, B_{j-1}} \\
				H_{i-1,j} + d \geq H_{i,j-1} + d
			\end{array} \right) \\
		B & \mbox{ako je} \left( \begin{array}{l} i=0 \\ j>0 \end{array} \right)
			\mbox{ili} \left( \begin{array}{l} i>0 \\ j>0 \\
				H_{i,j-1} + d > H_{i-1,j-1} + S_{A_{i-1}, B_{j-1}} \\
				H_{i,j-1} + d > H_{i-1,j} + d
			\end{array} \right) \\
	\end{array}
\right.
$$

Rekonstrukcija se, slično kao i kod Needleman-Wunschevog algoritma,
izvodi tako da krenemo iz polja $(t, u)$ i krećemo se  po matrici $R$ dok
ne dođemo do polja na kojem piše $X$ koje ovaj put ne mora nužno biti 
polje $R_{0,0}$. Dobiveni niz okrenemo i dobit ćemo traženo poravnanje.

\subsection{Primjer}
Za prethodno navedene ulazne podatke, matrice $H$ i $R$ izgledat će ovako:

$$
H =
\begin{tabular}{c|cccccccc}
 & - & T & G & C & A & C & T & A \\\specialrule{1pt}{0pt}{0pt}
- & 0 & 0 & 0 & 0 & 0 & 0 & 0 & 0 \\ 
A & \cellcolor{lightgray} 0 & 0 & 0 & 0 & 10 & 5 & 0 & 10 \\ 
T & 0 & \cellcolor{lightgray} 9 & 4 & 0 & 5 & 0 & 14 & 9 \\ 
G & 0 & 4 & \cellcolor{lightgray} 16 & 11 & 6 & 1 & 9 & 9 \\ 
C & 0 & 0 & 11 & \cellcolor{lightgray} 24 & \cellcolor{lightgray} 19 & 14 & 9 & 4 \\ 
C & 0 & 0 & 6 & 19 & 19 & \cellcolor{lightgray} 27 & 22 & 17 \\ 
G & 0 & 0 & 7 & 14 & 14 & \cellcolor{lightgray} 22 & 22 & 17 \\ 
T & 0 & 9 & 4 & 9 & 10 & 17 & \cellcolor{lightgray} 31 & 26 \\ 
A & 0 & 4 & 0 & 4 & 19 & 14 & 26 & \cellcolor{lightgray} 41 \\ 
\end{tabular}
$$

$$
R =
\begin{tabular}{c|cccccccc}
 & - & T & G & C & A & C & T & A \\\specialrule{1pt}{0pt}{0pt}
- & X & X & X & X & X & X & X & X \\ 
A & \cellcolor{lightgray} X & X & X & X & O & B & B & O \\ 
T & X & \cellcolor{lightgray} O & B & X & A & A & O & B \\ 
G & X & A & \cellcolor{lightgray} O & B & B & O & A & O \\ 
C & X & X & A & \cellcolor{lightgray} O & \cellcolor{lightgray} B & O & B & O \\ 
C & X & X & A & O & O & \cellcolor{lightgray} O & B & B \\ 
G & X & X & O & A & O & \cellcolor{lightgray} A & O & O \\ 
T & X & O & B & A & O & A & \cellcolor{lightgray} O & B \\ 
A & X & A & O & A & O & B & A & \cellcolor{lightgray} O \\ 
\end{tabular}
$$

Prema tome, traženo poravnanje jest $A' = \mbox{"TGC-CGTA"}$ i
$B' = \mbox{"TGCAC-TA"}$.

\subsection{Analiza složenosti}
Trivijalno je vidljivo da su memorijska i vremenska složenost i ovog algoritma 
jednake $O(NM)$. 


\section{Procjepi}
Do sada smo razmatrali samo slučaj kada je cijena otvaranja procjepa i
njegova proširivanja jednaka ($d$). Taj slučaj nazivamo
\textit{linearnom ocjenom procjepa},
budući da, ako je $k$ dužina procjepa, $dk$ je cijena za taj procjep. 
U praksi se primjenjuju još dva načina ocjenjivana procjepa.

Jedan je da za svaki procjep, bez obzira na njegovu dužinu, platimo fiksnu
cijenu ($c$), a nazivamo ga \textit{konstantnom ocjenom procjepa}. 

Drugi je kombinacija prethodna dva načina: za svaki procjep plaćamo fiksnu
cijenu ($c$), ali za njegovo produženje plaćamo neku drugu cijenu ($d$).
Takvu funkciju ocjene procjepa nazivamo \textit{Afinom ocjenom procjepa}.
Prema tome, za procjep dužine $k$, platit ćem cijenu jednaku $c + (k-1)d$.
Kako je empirijski pokazano da je ta funkcija u biti parobala, ovakva je
ocjena ujedno i najpreciznija. Nažalost, ona otežava računanje matrice
$H$ (i $R$). Problem je najlakše riješiti tako da uvedemo dvije nove
matrice $H^A$ i $H^B$ koje će nam pomoći u računanju cijene procjepa.
Matrice ćemo onda računati na slijedeći način:

$$
H_{i,j} =
\left\{ \begin{array}{ll}
	0 & \mbox{ako je } i=0 \mbox{ ili } j=0 \\
	\max \left( \begin{array}{l}
		0 \\
		H_{i-1,j-1} + S_{A_{i-1}, B_{j-1}} \\
		H^A_{i-1, j} + c \\
		H^B_{i, j-1} + c
	\end{array} \right) & \mbox{ako je } i>0 \mbox{ i } j>0
\end{array} \right.
$$

$$
H^{A}_{i,j} =
\left\{ \begin{array}{ll}
	0 & \mbox{ako je } i=0 \mbox{ ili } j=0 \\
	\max \left( \begin{array}{l}
		0 \\
		H_{i-1,j-1} + S_{A_{i-1}, B_{j-1}} \\
		H^A_{i-1, j} + d \\
		H^B_{i, j-1} + c
	\end{array} \right) & \mbox{ako je } i>0 \mbox{ i } j>0
\end{array} \right.
$$

$$
H^{B}_{i,j} =
\left\{ \begin{array}{ll}
	0 & \mbox{ako je } i=0 \mbox{ ili } j=0 \\
	\max \left( \begin{array}{l}
		0 \\
		H_{i-1,j-1} + S_{A_{i-1}, B_{j-1}} \\
		H^A_{i-1, j} + c \\
		H^B_{i, j-1} + d
	\end{array} \right) & \mbox{ako je } i>0 \mbox{ i } j>0
\end{array} \right.
$$

Rekonstrukcijska matrica vrlo je slična onoj u izvorno opisanom
algoritmu, a kako ćemo u udućem potpoglavlju maknuti potrebu za njom,
nećemo je posebno navoditi. 

\section{Memorijska optimizacija}
S obzirom da nizovi nukleotida mogu sadržavati i do nekoliko milijuna
elemenata, matrica dimenzija $N M$ fizički ne stane u memoriju. 
Međutim, čim primjetimo da za potrebe generiranja matrice $H$ (a ujedno i
$H^A$ odnosno $H^B$) koristimo
samo polja iz trenutnog i prethodnog retka, postaje jasno da nam ostali 
reci koje smo obradili ranije njih nisu potrebni. Stoga možemo pamtiti 
samo ta dva retka i time korištenu memoriju za proračun
matrice $H$ smanjimo na samo $O(\min(N, M))$. Ipak, problem se javlja
kada trebamo rekonstruirati rješenje pomoću matrice $R$, budući da
se prilikom rekonstrukcije potencijalno vraćamo se kroz sva njena polja. U
nastavku ćemo opisati rješenje tog problema.

-TODO: slika koja prikazuje da možemo koristiti samo dva posljednja retka

\subsection{Hirschbergov algoritam}
Dan Hirschberg osmislio je algoritam koji rješava problem rekonstrukcije
globalnog poravnanja u $O(NM)$ vremena koristeći $O(\min(N, M))$ memorije.
Njegova ideja temeljena je na metodi \textit{podijeli-pa-vladaj}.

No, prije nego je objasnimo, promotrimo što će se dogoditi s globalnim
poravnanjem ako okrenemo nizove $A$ i $B$. Okrenute nizove nazivat ćemo
$A^R$ i $B^R$, a matrice $F^R$ i $R^R$. Očito je da se ocjena poravnanja
neće  promijeniti, a prilikom rekonstrukcije dobit ćemo obrnuti niz. Uz sitne
modifikacije, možemo izračunati matricu $F^R$ bez da fizički okrećemo
nizove. 

Hirschbergov algoritam radi na slijedeći način:
\begin{enumerate}
\item
Bez smanjenja općenitosti, možemo pretpostaviti da je niz $A$ duži od
niza $B$. Neka nam $p$ označava polovicu dužine niza $A$, zaokruženu na dolje.
Podijelimo niz $A$ na dvije polovice, dužine $p$.

\item
Na obje polovice niza ($A_{0..p}$ i $A_{p..N}$) pokrenemo gore opisani algoritam
za pronalaženje
lokalnog poravnanja koje koristi $O(\min(N,M))$ memorije, ali bez
rekonstrukcije. Nad drugom polovicom niza radimo algoritam za obrnuto
nizove, tako da je za obje polovice posljednji izračunati red onaj odabran u
prethodnom koraku. 

\item
Definirajmo funkciju $\phi(j)$ kao sumu lokalnih poravnanja obje matrice u redu
$p$, tj. kao:
$$ \phi(j) = F_{p,j} + F^{R}_{p,j} $$
Barem će jedan član globalnog poravnanja, biti u retku $p$, a Hirschberg
je pokazao da će to biti onaj u stupcu $j$ za kojeg je $\phi(j)$ najveći.
Time smo pronašli jedan član koji je sigurno u globalnom poravnanju.

\item
S obzirom na svojstva globalnog poravnanja, intuitivno je jasno da
polja koja se nalaze gore-desno i dolje-lijevo od pronađenog polja 
sigurno nisu u točnom globalnom poravnanju, stoga ta polja
možemo izbaciti iz daljnjeg razmatranja. Ako rekurzivno primjenimo
algoritam na parove podnizova $A_{0..p}$ i $B_{0,j}$ te $A_{p..N}$ i
$B_{j..M}$, možemo ponavljajući postupak napraviti rekonstrukciju
cijelog niza.
\end{enumerate}

Ono što možda nije jasno jest vremenska složenost gore opisanog algoritma.
Pokazuje se da pri odbacivanju polja odbacimo u pravlu polovicu trenutno
razmatranih polja. Prema tome, broj operacija koje napravimo možemo napisati
u obliku geometrijskog niza:
$$ NM + NM/2 + NM/4 + ... < 2 NM \in O(NM) $$
Time dolazimo do zaključka da je složenost i dalje ostala jednaka do
razlike u konstantni, ali je znanto smanjena količina iskorištene
memorije, stoga se ova ušteda isplatiti koristiti.



- TODO: slike svih koraka


\section{Substitucija substitucijske matrice}
- TODO: neko bolje ime za ovo potpoglavlje

U ovom ćemo radu umjesto supstitucijske matrice za usporedbu dva elementa
koristiti fizičku udaljenost između dva atoma. Stoga će nizovi A i B biti
zapravo nizovi koordinata iz kojih ćemo moći za svaki par elemenata
izračunati koordinate. 


\chapter{Algoritam simuliranog kaljenja}
- objasni i zašto si uzeo više nizova odjednom, kako od njih biraš najboljeg i slično.
- cca 4 starnice

\chapter{CUDA tehnologija}
- cca 2 stranice

- prednosti i mane, problemi i rjesenja

\chapter{Implementacija}
- cca 5 stranica

- kako je sve spojeno

- detalji implementacije - koje funkcije ocjene sam koristio, kako sam došao do njih

\chapter{Rezultati}
- cca 1-2 starnice

\chapter{Zaključak}
Zaključak.
- cca 1-2 stranice

\bibliography{literatura}{}
\bibliographystyle{fer}
% cca 2 stranice

\begin{sazetak}
Sažetak na hrvatskom jeziku.

\kljucnerijeci{Ključne riječi, odvojene zarezima.}
\end{sazetak}

% TODO: Navedite naslov na engleskom jeziku.
\engtitle{Title}
\begin{abstract}
Abstract.

\keywords{Keywords.}
\end{abstract}

\end{document}
