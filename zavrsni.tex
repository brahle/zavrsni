\documentclass[times, utf8, zavrsni]{fer}
\usepackage{booktabs}
\usepackage{hyperref}
\hypersetup{
unicode=true, % za unicode podršku u bookmarkovima i sl., svakako postaviti na true
colorlinks=true, % za prikaz linkova u boji umjesto u pravokutnicima; slijede postavke po klasama linkova
linkcolor=black, % možete koristiti ime bilo koje definirane boje (pročitati post u kojem je predstavljen paket xcolor)
citecolor=black,
filecolor=black,
urlcolor=black,
linktocpage=false, % ako želite da oznaka stranice u tablici sadržaja bude link umjesto naslova, postavite na true
bookmarksopen=true, % pri otvaranju dokumenta će se prikazati toolbar s bookmarkovima, jako korisno
bookmarksopenlevel=1, % bookmarkovi su prikazani u stablastoj strukturi, a ovim određujete do koje će dubine stablo bookmarkova biti razgranato
bookmarksnumbered=true, % uz imena bookmarkova će se prikazati i njihov broj
pdftitle={SWIG - Poravnanje struktura korištenjem iterativne primjene Smith-Waterman algoritma},
pdfauthor={Bruno Rahle},
pdfsubject={tema},
pdfkeywords={kljucne rijeci - odvojite ih razmakom},
pdfstartpage={}, % ovisno koji stil numeriranja stranica koristite (\Roman, \roman, \arabic, \alph, \Alph), možete postaviti broj (oznaku) stranice koja će se prikazati prilikom otvaranja dokumenta; prazne vitičaste zagrade uvijek znače prvu stranicu, bez obzira na koji način bila numerirana
pdfstartview=FitH, % dokument će se prikazati tako da će mu širina odgovarati širini PDF preglednika
}


\begin{document}

% TODO: Navedite broj rada.
\thesisnumber{2321}

% TODO: Navedite naslov rada.
\title{SWIG - Poravnanje struktura korištenjem iterativne primjene Smith-Waterman algoritma}

% TODO: Navedite vaše ime i prezime.
\author{Bruno Rahle}

\maketitle

% Ispis stranice s napomenom o umetanju izvornika rada. Uklonite naredbu \izvornik ako želite izbaciti tu stranicu.
\izvornik

% Dodavanje zahvale ili prazne stranice. Ako ne želite dodati zahvalu, naredbu ostavite radi prazne stranice.
\zahvala{}

\tableofcontents

\listoffigures

\listoftables

\chapter{Uvod}
Problemi s kojima se znanost danas susreće često prelaze granice isključivo
jedne discipline. 

- pričaj o: bioinformatici, koje probleme ona rješava, zašto je problem koji si rješavao bitan,
kome koristi, kako se može koristiti i potencijalno za što se sve može koristiti.
- napiši zašto je ovo što si napravio jebeno i kako se može još poboljšati.
- cca 2 stranice

\chapter{Poravanavanje sturktura}
- detaljan opis problema
- cca 1-2 stranice

\chapter{Smith-Watermanov algoritam}
(cca 4-5 stranica)

Smith-Watermanov algoritam služi nam da bismo pronašli lokalno poravnanje. 
Osmislili su ga Temple F. Smith i Michael S. Waterman 1981. \cite{smithwaterman1981}.
Temelji se na Needleman-Wunschevom algoritmu (\cite{needlemanwunsch1970})
te i sam spada u kategoriju algoritama dinamičkog programiranja. 

- napisi jos teksta ovdje.

\section{Needleman-Wunschov algoritam}
Algoritam, kao što je već rečeno, traži globalno poravnanje. To znači da se svi 
članovi ulaznih nizova moraju poravnati. Dopuštene operacije kada tražimo poravnanje
su preklapanje s elementom iz suprotnog niza i ubacivanje praznina u neki od nizova.
Sve parove elemenata u dobivenom preklapanju bodujemo i na osnovu te ocjene
određujemo sličnost nizova. Konačan rezultat ovog algoritma jest poravnanje koje
maksimizira takvu ocjenu, tj. daje maksimalno globalno poravnanje.

Zanimljivost je da je to prvi algoritam dinamičkog programiranja ikada primjenjen
u bioinformatici.

\subsection{Ulazni podaci}
\begin{enumerate}
	\item Dva niza ($A$ i $B$) proteina ili nukleotida. Zbog jednostavnosti, pretpostavit
		ćemo da su to nukleotidi iz DNK - adenin (A), timin (T), gvanin (G) i citozin (C).
		U primjeru ćemo koristiti $A = "ATGCCGTA"$ i $B = "TGCCTA"$. Dužinu niza $A$ označit
		ćemo s $N$, a dužinu niza $B$ s $M$. 
	\item Matrica $S$, koja nam daje bodove koje dobijemo kada jedan nukletoid preklopimo
		s drugim. U našem će slučaju imati dimenzije $4 \times 4$, budući da ćemo razmatrati
		slučaj kada imamo samo četiri nukleotida. U principu će na dijagonali imati pozitivne
		brojeve, a na ostalim poljima negativne. To znači da nam se najviše isplati 
		preklapati nukletide istog tipa, jer za to dobivamo bodove, a inače
		gubimo bodove. Primjer jedne takve matrice koju ćemo koristiti i u primjeru:
		
		$\begin{tabular}{c|cccc}
		& A & C & G & T \\\specialrule{1pt}{0pt}{0pt}
		A & 10 & -3 & -9 & -1 \\
		C & -5 & 8 & -8 & -7 \\
		G & -5 & -4 & 7 & -5\\
		T & -4 & -11 & -8 & 9
		\end{tabular}
		$
	\item Negativan broj $d$, koji označava bodove koje dobijemo (tj. izgubimo) kada nukleotid
		preklopimo s prazninom. 
\end{enumerate}

\subsection{Izlazni podaci}
\begin{enumerate}
	\item Broj $H$, ocjena najboljeg globalnog poravnanja.
	\item Dva nova niza jednake dužine, $A'$ i $B'$, nastala ubacivanjem praznina
		(označenih najčešće sa '-') u nizove $A$ i $B$ koja predstavljaju najbolje pronađeno
		poravnanje.
\end{enumerate}

\subsection{Algoritam}
Neka nam matrica $F_{i,j}$ označava maksimalan broj bodova koje možemo dobiti kada
poravnamo prvih $i$ članova niza $A$ i prvih $j$ članova niza $B$. $F_{i,j}$ možemo
računati rekurzijom na slijedeći način:
$$
F_{i,j} = \left\{
	\begin{array}{lr}
		0 & \mbox{ako je } i=0 \mbox{ i } j=0 \\
		i \cdot d & \mbox{ako je } i>0 \mbox{ i } j=0 \\
		j \cdot d & \mbox{ako je } i=0 \mbox{ i } j>0 \\
		max \left(
			\begin{array}{l}
				F_{i-1,j-1} + S_{A_{i-1}, B_{j-1}} \\
				F_{i-1, j} + d \\
				F_{i, j-1} + d
			\end{array}
		\right) & \mbox{ako je } i>0 \mbox{ i } j>0
	\end{array}
\right.
$$

\subsection{Primjer}

\subsection{Analiza složenosti}


\chapter{Algoritam simuliranog kaljenja}
- objasni i zašto si uzeo više nizova odjednom, kako od njih biraš najboljeg i slično.
- cca 4 starnice

\chapter{CUDA tehnologija}
- cca 2 stranice

- prednosti i mane, problemi i rjesenja

\chapter{Implementacija}
- cca 5 stranica

- kako je sve spojeno

- detalji implementacije - koje funkcije ocjene sam koristio, kako sam došao do njih

\chapter{Rezultati}
- cca 1-2 starnice

\chapter{Zaključak}
Zaključak.
- cca 1-2 stranice

\bibliography{literatura}{}
\bibliographystyle{fer}
% cca 2 stranice

\begin{sazetak}
Sažetak na hrvatskom jeziku.

\kljucnerijeci{Ključne riječi, odvojene zarezima.}
\end{sazetak}

% TODO: Navedite naslov na engleskom jeziku.
\engtitle{Title}
\begin{abstract}
Abstract.

\keywords{Keywords.}
\end{abstract}

\end{document}
